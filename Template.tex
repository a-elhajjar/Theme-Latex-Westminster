\documentclass[14pt,xcolor=svgnames]{beamer} %Beamer
% You can change the point size for the fonts. 
\usepackage{palatino} %font type
\usepackage{xcolor,colortbl}


\usefonttheme{metropolis} %Type of slides
\usefonttheme[onlymath]{serif} %font type Mathematical expressions
\usetheme[progressbar=frametitle,titleformat frame=smallcaps,numbering=counter]{metropolis} %This adds a bar at the beginning of each section.
\useoutertheme[subsection=false]{miniframes} %Circles in the top of each frame, showing the slide of each section you are at
\setbeamertemplate{blocks}[rounded]
\setbeamercolor{block title}{use=structure,fg=black,bg=bg2!90}
\setbeamercolor{block body}{parent=normal text,use=block title,bg=bgu!90}
\setbeamercolor{block title alerted}{use=structure,fg= blck, bg= bgu!90}
\setbeamercolor{block body alerted}{parent=normal text,use=block title,fg= black, bg=bg2!90}

%You can do the same for all other blocks
\usepackage{tcolorbox}
\usepackage{enumitem}

\usepackage{appendixnumberbeamer} %enumerate each slide without counting the appendix
\definecolor{blck}{RGB}{124,0,63}
\definecolor{bgu}{RGB}{33,138,134}
\definecolor{bg2}{RGB}{255,223,0}
\definecolor{sep}{RGB}{138,169,187}
\setbeamercolor{progress bar}{bg=bgu,fg=blck} %These are the colours of the progress bar. Notice that the names used are the svgnames
\setbeamercolor{title separator}{fg=blck} %This is the line colour in the title slide
\setbeamercolor{structure}{bg=bg2, fg=white} %Colour of the text of structure, numbers, items, blah. Not the big text.
\setbeamercolor{normal text}{bg=bgu, fg=white} %Colour of normal text
\setbeamercolor{alerted text}{fg=sep} %Color of the alert box
\setbeamercolor{example text}{fg=Maroon!70!Coral} %Colour of the Example block text
\setbeamercolor{palette primary}{fg=black, bg=bg2} %These are the colours of the background. Being this the main combination and so one. 
\setbeamercolor{palette secondary}{bg=bg2, fg=black}
\setbeamercolor{palette tertiary}{bg=bg2, fg=black}
\setbeamercolor{section in toc}{bg=bg2, fg=white} %Color of the text in the table of contents (toc)

\usepackage{pifont}


%These next packages are the useful for Physics in general, you can add the extras here. 
\usepackage{amsmath,amssymb}
\usepackage{slashed}
\usepackage{cite}
\usepackage{relsize}
\usepackage{caption}
\usepackage{subcaption}
\usepackage{multicol}
\usepackage{booktabs}
\usepackage[scale=2]{ccicons}
\usepackage{pgfplots}
\usepgfplotslibrary{dateplot}
\usepackage{geometry}
\usepackage{xspace}
\newcommand{\themename}{\textbf{\textsc{bluetemp}\xspace}}%metropolis}}\xspace}
\usepackage{float}
%Figure positioning. I use tikz but you can use other.
\usepackage{tikz}
\usepackage{graphicx}
\usetikzlibrary{calc}
\title{Title of presentation}
\author[Short name]{Presenter name} %With inst, you can change the institution they belong
\subtitle{Subtitle}
\institute[uni]{ School of XXXXXXX  \\ University of Westminster}
\date{\today} %Here you can change the date. if you want to change to a future date use the date instead of \today for example - \date{October 01, 2022}
\titlegraphic{\vspace{-0.5cm}\hfill\includegraphics[scale=0.05]{logo.png}} %You can modify the location of the logo by changing the command \vspace{}. 

\begin{document}
{
\setbeamercolor{background canvas}{bg=bgu, fg=black}
\setbeamercolor{normal text}{fg=white}
\maketitle
}%This is the colour of the first slide. bg= background and fg=foreground

\metroset{titleformat frame=smallcaps} %This changes the titles for small caps

\begin{frame}{Outline}
\setbeamertemplate{section in toc}[sections numbered] %This is numbering the sections
\tableofcontents[hideallsubsections] %You can comment this line if you want to show the subsections in the table of contents
\end{frame}

\begin{frame}{Objectives}
\underline{\textsc{Some text:}}
\newline
\textbf{\textsc{Some text:}}
\begin{small}
This is some small Text. 
\end{small} 
\newline
\begin{Large}
	This is some Large Text. 
\end{Large}
\newline
\begin{Huge}
	This is some Huge Text. 
\end{Huge}

\end{frame}


\section{Introduction}

\begin{frame}[fragile]{Introduction: blah blah} %You can change fragile by standout

\begin{itemize} %The symbol of the items can be changed by which ever you want, this is just an example. using pifont package for sympols. full list of codes is here https://latex-tutorial.com/bullet-styles/
 \item[\ding{51}]Text
	\item[\ding{56}] Text
	\item[\ding{43}] Text
	\item[\ding{118}] Text
	\item[\ding{48}] Text

\end{itemize}
TextTextText
TextTextText
\begin{enumerate}[label=(\Alph*)]
	\item Text
\begin{enumerate}[label=\Roman*.]
	\item Text
	\item Text
\end{enumerate}	
\begin{enumerate}[label=\arabic*.]
	\item Text
	\item Text
\end{enumerate}	
\end{enumerate}

\end{frame}

\begin{frame}[standout]{This is other type of slide}

An equation without number could be represented by:
\begin{equation*}
	c^{2} = a^{2} + b^{2}
\end{equation*}
And an equation with number:
\begin{equation}
E^{2} = m^{2} + p^{2}
\end{equation}
\end{frame}


\section{This is another section}
\begin{frame}{Frame Title} %You can also not write fragile or standout and you can see how it looks
Hello world!
\end{frame}
\section{Blocks}
	\begin{frame}{Blocks colours}
		
		\begin{block}{Block 1}
			\begin{itemize}
			\item one
			\item two
			\item three
			\end{itemize}
		\end{block}
		

		\begin{alertblock}{Block alert}
			\begin{itemize}
				\item example
			\end{itemize}
		\end{alertblock}
	\end{frame}


\begin{frame}{Blocks colours (Cont.)}
	

	
	\begin{definition}{Block definition}
		\begin{itemize}
			\item example
		\end{itemize}
	\end{definition}
	
	\begin{theorem}{Block theorem}
		\begin{itemize}
			\item example
		\end{itemize}
	\end{theorem}
\end{frame}

\section{Tables and figures}
	
	\begin{frame}{Figures}
		
		% Commands to include a figure. This is centered. 
		\begin{figure}
		\includegraphics[width=0.2\textwidth]{placeholder}
		\caption{\label{fig:default_positioning}default position.}
		\end{figure}
		% Commands to include a figure at a specific location
		\begin{figure}  
		\begin{tikzpicture}[remember picture,overlay]
			\node[anchor=south west,inner sep=0pt] at ($(current page.south west)+(2cm,2cm)$) {

		\includegraphics[width=0.2\textwidth]{placeholder}


	};
\end{tikzpicture}


		\end{figure}
	\end{frame}
	
\begin{frame}{Table}
	
	\begin{table}
		\centering
		\begin{tabular}{l|r}
			One & Two \\\hline
			One & 42 \\
			Two & 13
		\end{tabular}
		\caption{\label{tab:1}An example table.}
	\end{table}

\begin{table}
	\centering
	\begin{tabular}{>{\columncolor{bg2}|}l r|}
	One & Two \\\hline
			One & 42 \\
	\end{tabular}
	\caption{\label{tab:2}An example table.}
\end{table}
\begin{table}
	\centering
	\begin{tabular}{>{\columncolor{bg2}|}l >{\columncolor{blck}|}r|}
		\rowcolor{bgu}	One & Two \\\hline
		One & 42 \\
	\end{tabular}
	\caption{\label{tab:3}An example table.}
\end{table}
	\end{frame}
\section{Final section}


\begin{frame}{Conclusion}
These are the final words, you do your best to try to wake up everyone that was listening to your talk.
\end{frame}

\section{Final Notes}


\begin{frame}{Footnotes}
 final word \footnote{\href{overleaf}{https://www.overleaf.com/latex/templates/beamer/ybffdrynfwjp.pdf}}
\end{frame}
\begin{frame}[standout]
Text
\end{frame}

\appendix

\begin{frame}{Back up}
Text
\end{frame}

\end{document}